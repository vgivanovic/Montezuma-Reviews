%%% Time-stamp: <2024-09-22 13:52:46 vladimir>
%%% Copyright (C) 2019-2024 Vladimir G. Ivanović
%%% Author: Vladimir G. Ivanović <vladimir@acm.org>
%%% ORCID: https://orcid.org/0000-0002-7802-7970

\chapter[Glossary]{\centering\normalfont\normalsize GLOSSARY}\label{ch:glossary}
\begin{description}[nosep]%

  \medskip\item[ADA] Average Daily Attendance, the method that the state of California uses to determine how many students attend a particular school. An alternative is to use the number of students enrolled, some of whom may attend sporadically but still need to be educated when they do attend. \parencite{SACS2019}

  \medskip\item[arm's length transaction] A transaction, often financial, where all parties are independent and self-interested. \parencite{Wex2024}

  \medskip\item[blended learning] A method of teaching where both in-person

  instruction and virtual instruction are used. \parencite{Graham2018}

  \medskip\item[bond] A bond is a loan whose terms (maturity date, interest rate) are fixed. Bonds are issued by a borrower (the debtor) to investors (the creditors) who are the source of the funds borrowed. The borrower is liable for repaying the debt, usually on a fixed schedule. In return for getting the funds now, the borrower agrees to compensate the creditor by repaying both the amount loaned (the principal) and interest on the amount outstanding at an agreed upon (ther interest) rate. \parencite{Borad2015}

  \medskip\item[charter school] A publicly funded but privately run school that is independent from the usual public school system. Like public schools, charter schools are tuition-free, must accept all who apply, and are governed by a school board. Unlike public schools, the school board is unelected. \parencite{CDE2023, CSBA2016, Eckes2024}

  \medskip\item[charter school authorizer] A governmental entity that grants charter schools the authority to operate and which provides oversight. In California, a chartering authority could be a public school district, a county office of education, or the California Department of Education. \parencite{NACSA2024}

  \medskip\item[charter management organization (CMO)] ``A non-profit organization that operates or manages a network of charter schools (either through a contract or as the charter holder) linked by centralized support, operations, and oversight.'' \parencite[2]{USDoEd2021} and \parencite{CDE2023a}.

  \medskip\item[conduit bond] A type of municipal bond where the bond is paid back, not by a public entity's revenue stream, but by a private entity, the conduit borrower, e.g. a limited liability company or corporation. The public entity, the conduit issuer, is merely a pass through entity between investors and the conduit borrower \parencite{Cooper2017}. (See \textcite{GASB91_2019} for details on what qualifies as a conduit bond.) %chktex 8

  \medskip\item[cross-collateralization] A term from bond financing which indicates that an asset has been used as collateral in two different obligations \parencite{Lip2024}.

  \medskip\item[debt, convertible] An obligation (a loan or a bond) that can be converted into another form, typically common stock or equity \parencite{Chen2020}, but in Rocketship's case, from a loan into a grant or donation.

  \medskip\item[double bottom line grantors] Grantors (philanthropies) which measure social impact in addition to fiscal performance \parencite{Clark.etal2015}.

  \medskip\item[education management organization (EMO)] ``A for-profit entity that operates or manages a network of charter schools (either through a contract or as the charter holder) linked by centralized support, operations, and oversight.'' \parencite[2]{USDoEd2021} and \parencite{CDE2023a}. %chktex 38

  \medskip\item[general obligation bonds (GO)] General obligation bonds are tax-exempt bonds backed by a public entity's tax revenues, and not from the revenue of a project. California state law limits bond debt to 2.5\% of total assessed valuation for unified school districts and 1.25\% for elementary and high school districts \parencite{CDIAC2014}.

  \medskip\item[municipal bond] A municipal bond is a bond issued by a public entity and bought by investors. The public entity (the debtor) borrows from investors (the creditor). Investors loan money to the public entity, and the public entity pays the investors back over time with interest \parencite{Chen2022}.

  \medskip\item[parcel tax] A non-\textit{ad valorem} property tax, i.e.~not based on the value of the property, but assessed per parcel \parencite{Lu2019}.

  \medskip\item[philanthrocapitalism] Using a market capitalism approach in non-profits \parencite{Giridharadas2018}.

  \medskip\item[portfolio school district] A collection of diverse charter schools managed as a single organization \parencite{Lake.Hernandez2011}.

  \medskip\item[property tax] A tax based on the assessed value of a property, i.e.~an \textit{ad valorem} tax \parencite{BOE2018}.

  \medskip\item[Proposition 13] Passed by California voters in 1978 as a constitutional amendment, Prop. 13 devastated funding to local governments, including school districts by limiting the property tax to 1\% of assessed value, increases to a maximum of 2\% (unless reassessed because of a change in ownership), and requiring a two-thirds majority to increase property taxes, since lowered to 55\% by Prop. 39 \parencite{Aguinaldo.etal2022}.

  \medskip\item[Proposition 39] Passed by California voters in 2000 as a constitutional amendment and state statute, Prop. 39 mandates that public school districts, if requested, must provide reasonably equivalent facilities to charter schools \parencite{Aguinaldo.etal2022}.

  \medskip\item[Proposition 98] Passed by California voters 1988 as a constitutional amendment and state statute, Prop. 98 defines the minimum funding level of K-14 schools in the state budget \parencite{Aguinaldo.etal2022}.

  \medskip\item[public school] Public schools are funded by taxpayers and are governed by a publicly elected Board of Trustees. Unlike charter schools, public schools accept, at any time of year, any and all students who wish to enroll. They do not discriminate on the basis of race, national origin, sexual orientation, gender, religion, citizenship, ability, disability, or language proficiency. All are welcome \parencite{CDE2023b}. (See also Jamie Vollmer's story about blueberries in \textcite{Vollmer2011}.)

  \medskip\item[related party transaction] A transaction that is not an ``arm's length transaction'' \parencite{Kenton2022}.

  \medskip\item[revenue bonds] Tax-exempt bonds guaranteed by a schools revenue instead of by an LEA's property tax revenue \parencite{Chen2021}.

  \medskip\item[unduplicated pupils] The State of California augments school district revenue on a per pupil basis for every pupil that qualifies for free or reduced price lunch, or is an English language learner, or is a foster youth, but only an unduplicated basis. Notably, children with special needs are not considered \textit{unduplicated pupils}. Neither are homeless children \parencite{CDE2015}.

\end{description}

%%% Local Variables:
%%% mode: latex
%%% TeX-master: "Rocketship_Education-An_Exploratory_Public_Policy_Case_Study"
%%% End:
